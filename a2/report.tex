\documentclass{article}
\usepackage{amsfonts}

\usepackage{amsmath}
\DeclareMathOperator*{\argmax}{arg\,max}
\DeclareMathOperator*{\argmin}{arg\,min}

\usepackage{cite}
\usepackage[utf8]{inputenc}
\usepackage{listings}
\usepackage{graphicx}
\usepackage[ruled,vlined]{algorithm2e}
\usepackage{color}
 
\definecolor{codegreen}{rgb}{0,0.6,0}
\definecolor{codegray}{rgb}{0.5,0.5,0.5}
\definecolor{codepurple}{rgb}{0.58,0,0.82}
\definecolor{backcolour}{rgb}{0.95,0.95,0.92}
 
\lstdefinestyle{mystyle}{
    backgroundcolor=\color{backcolour},   
    commentstyle=\color{codegreen},
    keywordstyle=\color{magenta},
    numberstyle=\tiny\color{codegray},
    stringstyle=\color{codepurple},
    basicstyle=\footnotesize,
    breakatwhitespace=false,         
    breaklines=true,                 
    captionpos=b,                    
    keepspaces=true,                 
    numbers=left,                    
    numbersep=5pt,                  
    showspaces=false,                
    showstringspaces=false,
    showtabs=false,                  
    tabsize=2
}
 
\lstset{style=mystyle}

\title{IE 613: Assignment 1}
\author{Manan Doshi \\ 140100015}
\date{20 February 2018}

\begin{document}
\graphicspath{{./plots}}
\maketitle

\section*{Question 1}

In this question, we show that the Weighted Majority algorith we developed for the case of a finite number of experts is a special case of the FoReL algortihm when:
\begin{align*}
    R_w  &= \frac{1}{\eta}\sum_{j=1}^{d} w_j \log w_j \qquad &\text{$R_w$ is the regularizer}\\
    f_t(w) &= \langle w,v_t \rangle \qquad &\text{$f_t(w)$ is the linear loss function}\\
    \sum_{j=1}^{D} w_j &= 1 \qquad &\text{$w$ does not span the entire space $\mathbb{R}_d$}\\
\end{align*}

We show that this is analogous to the WMA with $D$ experts where $v_t$ is the loss vector associated with round $t$ and $w_t$ is the (regularized) weight of each expert.\\
For the FoReL algorithm:
\begin{align*}
    w_t &= \argmin_{w} \left( \sum_{i=1}^{t-1} f_i(w) + R(w) \right) &\qquad\\
    &= \argmin_{w} \left( \sum_{i=1}^{t-1} \langle w,v_i \rangle + \frac{1}{\eta} \sum_{j=1}^{D} w_j \log w_j \right) &\qquad \text{Substituting regularizer and loss terms}\\
\end{align*}

We need to find the optimal value of $w$ under the constraint $\sum_j w = 1$. We use the lagrange multiplier method.
\begin{align*}
    l(w) &= \left( \sum_{i=1}^{t-1} \langle w,v_i \rangle + \frac{1}{\eta} \sum_{j=1}^{D} w_j \right)\ &\qquad \text{We wish to minimise this function}\\
    c(w) &= \sum_{j=1}^{D} w_j - 1 &\qquad \text{under the constraint that $c=0$}\\
    \nabla_w(l(w)) &=\sum_{i=1}^{t-1} v_i + \frac{1}{\eta}\left( \bf{1} + \log w \right) \\
    \nabla_w(c(w)) &= \bf{1}\\
    \mathcal{L}(w) &= l(w) - \lambda c(w) \\
    \nabla_{w,\lambda}\mathcal{L}(w,\lambda) &= 0  &\qquad \text{Lagrange multiplier method}\\
    \sum_{i=1}^{t-1} v_i + \frac{1}{\eta}\left( \bf{1} + \log w \right) &= \lambda(\bf{1})
\end{align*}
Solving for $w$,
\begin{align*}
    w_t &= \exp (\eta \lambda - 1) \exp (-\eta \sum_{i=1}^{t-1}v_i)\\
    w_{j,t}    &= \frac{e^{-\eta \sum_{i=1}^{t-1}v_{i,j}}}{\sum_{j=1}^{D}e^{-\eta \sum_{i=1}^{t-1}v_{i,j}}} &\qquad \text{This is because we choose $\lambda$ such that the constraint is satisfied}\\
\end{align*}

This is the exact same expression obtained in WMA, where $v_{i,j}$ in the loss suffered by Expert $j$ in round $i$. The $\eta$ here is the same as the $\eta$ in WMA. The optimal $\eta$ will thus be

\[
    \eta^* = \sqrt{\frac{2 \log D}{T}}
    \]

\newpage
\section*{Question 2}
%
%\subsection*{Code}
%\begin{lstlisting}[language=python]
%import numpy as np
%import matplotlib.pyplot as plt
%
%def wma(d,T,eta):
%    w_tilde = np.ones([d])
%    l       = np.zeros([d,T])
%    loss    = 0
%    e_loss  = 0
%    for t in range(T):
%        w          = w_tilde/np.sum(w_tilde)
%        adv_choice = np.random.choice(d,p=w)
%        l[:-2,t]   = np.random.choice(2, size=8, p=[0.5, 0.5])
%        l[-2,t]    = np.random.choice(2, p=[0.6,0.4])
%        delta      = 0.1 if t<T/2 else -0.2
%        l[-1,t]     = np.random.choice(2, p=[0.5-delta,0.5+delta])
%        loss      += l[adv_choice,t]
%        e_loss    += w.dot(l[:,t])
%        w_tilde    = w_tilde*np.exp(-eta*l[:,t])
%
%    costs    = np.sum(l,axis=1) 
%    regret   = loss - np.min(costs)
%    p_regret = e_loss - np.min(costs)
%    
%    return p_regret
%
%d = 10      #Number of advisors
%T = 100000  #Number of rounds
%
%c         = np.linspace(0.1,2.1,11)
%Eta       = c*np.sqrt(2.0*np.log(d)/T)
%n_samples = 30
%R         = np.zeros([11,n_samples])
%for i,eta in enumerate(Eta):
%    for trial in range(n_samples):
%        R[i,trial] = wma(d,T,eta)
%        print("Sample: {}, i_c:{}".format(trial,i))
%
%m, s   = np.mean(R, axis=1), np.std(R, axis=1, ddof=1)*1.96/np.sqrt(n_samples)
%
%fig,ax = plt.subplots(figsize=(15,15))
%ax.errorbar(c,m,s)
%ax.set_xticks(c)
%ax.tick_params(axis='both', labelsize=15)
%ax.set_xlabel(r"$\frac{\eta}{\sqrt{\frac{2\log(d)}{T}}}$", \\
%                fontsize=40, labelpad=20)
%ax.set_ylabel(r"Expected regret", fontsize=20, labelpad=30)
%plt.savefig(r"./plots/q1.png")
%plt.show()
%\end{lstlisting}
%\newpage
%\subsection*{Plots}
%\begin{figure}[h!]
%\centering
%\includegraphics[scale=0.22]{q1}
%\caption{Variation of expected regret with $\eta$ for the weighted majority algorithm}
%\end{figure}
%
%\begin{figure}[h!]
%\centering
%\includegraphics[scale=0.3]{q1b_b}
%\caption{Probability contours with advisors and time. Higher $\eta$ leads to more aggressive exploitation. The algorithm is not able to switch over to the better advisor after $T/2$ due to lack of exploration}
%\end{figure}
%
%\newpage
%
%\section*{Question 2}
%\subsection*{Code}
%\subsubsection*{EXP3}
%\begin{lstlisting}[language=python]
%def exp3(d,T,eta):
%    e_loss = 0
%    elv = 0.5*np.ones([d,2])
%    elv[-2,:] = 0.4
%    elv[-1,:] = [0.6,0.3]
%    w_tilde = np.ones([d])
%    
%    for t in range(T):
%        w = w_tilde/np.sum(w_tilde)
%        adv_choice = np.random.choice(d,p=w)
%        e_loss_c   = elv[adv_choice,(2*t)//T]
%        l          = np.random.choice(2,\
%                     p=[1-e_loss_c, e_loss_c])/w[adv_choice]
%        e_loss    += e_loss_c
%        w_tilde[adv_choice]    = w_tilde[adv_choice]*np.exp(-eta*l)
%
%    return e_loss - 0.4*T
%\end{lstlisting}
%\subsubsection*{EXP3.P}
%\begin{lstlisting}[language=python]
%def exp3p(d,T,eta,beta,gamma):
%    e_gain = 0
%    elv = 0.5*np.ones([d,2])
%    elv[-2,:] = 0.4
%    elv[-1,:] = [0.6,0.3]
%    egv = 1-elv
%    G = np.zeros([d])
%    w_tilde = np.ones([d])
%
%    for t in range(T):
%        w                 = (1-gamma)*(w_tilde/np.sum(w_tilde)) + gamma/d
%        adv_choice        = np.random.choice(d,p=w)
%        e_gain_c          = egv[adv_choice,(2*t)//T]
%        gain              = beta/w
%        gain[adv_choice] += np.random.choice(2,\
%                            p=[1-e_gain_c, e_gain_c])/w[adv_choice]
%        e_gain           += e_gain_c
%        w_tilde           = w_tilde*np.exp(eta*gain)
%
%    return 0.6*T - e_gain
%\end{lstlisting}
%\newpage
%\subsubsection*{EXP3-IX}
%\begin{lstlisting}[language=python]
%def exp3ix(d,T,eta,gamma):
%    e_loss = 0
%    elv = 0.5*np.ones([d,2])
%    elv[-2,:] = 0.4
%    elv[-1,:] = [0.6,0.3] 
%    w_tilde = np.ones([d])
%
%    for t in range(T):
%        w = w_tilde/np.sum(w_tilde)
%        adv_choice = np.random.choice(d,p=w)
%        e_loss_c   = elv[adv_choice,(2*t)//T]
%        l          = np.random.choice(2,\
%                     p=[1-e_loss_c, e_loss_c])/(w[adv_choice]+gamma)
%        e_loss    += e_loss_c
%        w_tilde[adv_choice]    = w_tilde[adv_choice]*np.exp(-eta*l)
%
%    return e_loss - 0.4*T
%\end{lstlisting}
%\subsection*{Plot}
%\begin{figure}[h!]
%\centering
%\includegraphics[scale=0.24]{Q2}
%\caption{Variation of expected regret with $\eta$ multiplier}
%\end{figure}
%\newpage
%\section*{Question 3}
%Clearly, \verb|EXP3-IX| has the best performance with lower expected regret and lower deviation. The good performance of \verb|EXP3-IX| can be attributed to the fact that it explores and detects the new best advisor after the change in odds at $\frac{T}{2}$. This exploration is not possible in \verb|EXP3|. The bad performance of \verb|EXP3.P| can be attributed to very high exploration rates leading to low exploitation of the current best adviser. This behavious can be clearly seen in the following plot
%
%\begin{figure}[h!]
%\centering
%\includegraphics[scale=0.3]{Q3}
%    \caption{Probability contours with advisors and time for \texttt{EXP3}, \texttt{EXP3.P} and \texttt{EXP3-IX} respectively. The weak exploitation of \texttt{EXP3.P} and the slow switching of \texttt{EXP3} is apparent here.}
%\end{figure}
%\newpage
%\section*{Question 4 \cite{ben2009agnostic}}
%The proposed algorithm is the same as the wighted majority algorithm
%
%\begin{algorithm}
%    \SetKwInOut{Input}{Input}
%    \SetKwInOut{Parameter}{Parameter}
%    \SetKwInOut{init}{Initialize}
%
%    \Input{Hypothesis class $\mathcal{H}$}
%    \Parameter{$\eta \in [0,1]$}
%    \init{$\tilde{w}^{(1)} = [1, 1, 1, ... ,1]$ in \(\mathbb{R}^d\)}
%    \For{$t\leftarrow 1$ \KwTo $T$}{
%        Set $w_i^{(t)} = \frac{\tilde{w}_i^{(t)}}{\sum_i \tilde{w}_i^{(t)}}$ \\
%        Play $i$ according to the distribution $w^{(t)}$ \\
%        Receive loss vector $l_t = \{l_{t,i} :\forall i \in d\}$ where $l_{t,i}$ is the error in prediction of hypthesis $h_i$\\
%        Update $\forall i, \tilde{w}_i^{(t+1)} = \tilde{w}_i^{(t)}e^{-\eta l_{t,i}}$\\
%    }
%    \caption{The Weighted Majority Algorithm}
%\end{algorithm}
%
%We will compute a finite bound for expected number of mistakes of this algorithm on a realizable case with Bernoulli noise.
%We first make the claim that
%\begin{equation}
%\label{main}
%\mathbb{E}\left[\sum_{s=t+1}^{T}\|\hat{y}_s-f_i^s\|\right]_{w_t} \leq C_{\gamma} \ln\left(\frac{Z_t}{w_i^t}\right)
%\end{equation}
%where $i$ refers to the `correct' hypothesis. $Z_t = \sum_{i=1}^d w_i^t$ and \(C_{\gamma} = \frac{1}{1-2\sqrt{\gamma(1-\gamma)}}\).
%
%	We will now prove the above claim using induction. The base case at $t=T$ is trivial since the LHS is $0$ and the right side is positive (since $Z_t \geq w_i^t$). We will now split our hypothesis class into two groups based on whether the hypothesis classifies the round at $t$ correctly.
%\[
%	u = \sum_{j, f_j^t = f_i^t} w_j^{t-1} \qquad\qquad  \qquad v = \sum_{j, f_j^t /neq f_i^t} w_j^{t-1}
%\]
%
%$u$ is thus the total weight of the correct classifiers for the round and $v$ is the total weight of the incorrect classifiers. Probability that the algorithm classifies incorrectly is thus $\frac{v}{Z_t-1}$. There is also a chance that the system sends incorrect feedback, say with probability $p \leq \gamma$. If the feedback is incorrect the weight update is $Z_t = e^{-\eta}u+v$ and the update of the weight of the correct hypothesis class is $ w_i^t = e^{\eta} w_i^{t-1} $. If the system send correct feedback (with probability \(1-p\)) and the weight of the correct hypothesis remains unchanged. Expected mistakes from $t$ to $T$ equals the expected number of mistakes at $t$ plus the expected number of mistakes from $t+1$ to $T$. This leads us to
%\begin{align*}
%	\mathbb{E}\left[\sum_{s=t}^{T}\|\hat{y}_s-f_i^s\|\right]_{w_{t-1}} &= \mathbb{E}\left[\sum_{s=t+1}^{T}\|\hat{y}_s-f_i^s\|\right]_{w_t} + \frac{v}{Z_{t-1}} \\
%&\leq \frac{v}{Z_{t-1}} + \mathbb{E}\left[C_{\gamma}\ln{\left(\frac{Z_t}{w_i^t}\right)}\right]_{w_t}\\
%&= \frac{v}{Z_{t-1}} + p\left[C_{\gamma}\ln{\left(\frac{e^{-\eta}u + v}{e^{-\eta}w_i^{t-1}}\right)}\right] +  (1-p)\left[C_{\gamma}\ln{\left(\frac{u + e^{-\eta}v}{w_i^{t-1}}\right)}\right]\\
%\end{align*}
%
%We will show that the last expression is bounded by the RHS of \eqref{main}. This involves mathematical manipulations given in the appendix of \cite{ben2009agnostic}. Once we have proved \eqref{main} we can substitute $t=0$ to get an upper bound for expectation of number of mistakes.
%
%\[
%\boxed{\mathbb{E}\left[\sum_{s=1}^{T}\|\hat{y}_s-f_i^s\|\right]_{w_t} \leq C_{\gamma} \ln(d)}
%\]
%\newpage
%\section*{Question 5}
%Consider an algorithm \texttt{A} whose regret bound for $T$ rounds is $\alpha \sqrt{T}$. For $2^m$ rounds, the regret bound will be $\alpha \sqrt{2^m}$. Since we do not know the time horizon, we break the time period into pieces of size $2^m$ where $m = 0, 1, 2, \cdots$. We choose the parameter $\eta$ in terms of these smaller time periods for every packet.
%
%If the total time horizon is $T$ and the total number of `packets' is $k$,
%\begin{align*}
%    \sum_{m=0}^{k-1} 2^m +1              &&\leq T \leq &&\sum_{m=0}^{k} 2^m \\
%    1 + 1 + 2 + 2^2 + 2^3 \cdots + 2^k   &&\leq T \leq &&1 + 2 + 2^2 + 2^3 \cdots + 2^k\\
%    2^k                                  &&\leq T \leq &&(2^{k+1}-1)
%\end{align*}
%
%For a time period of $2^m$, regret is $\alpha 2^{\frac{m}{2}}$. Total regret:
%\begin{align*}
%    \mathcal{R} &\leq \sum_{m=0}^{k} \alpha 2^{\frac{m}{2}}\\
%                &\leq \alpha \left(1 + \sqrt{2} + \sqrt{2}^2 + \cdots + \sqrt{2}^k \right) \\
%                &\leq \alpha \left( \frac{2^{\frac{k+1}{2}} - 1}{\sqrt{2}-1}\right)\\
%                &\leq \alpha \left( \frac{\sqrt{2T} - 1}{\sqrt{2}-1}\right)\\
%                &\leq \alpha \left( \frac{\sqrt{2T}}{\sqrt{2}-1}\right)
%\end{align*}
%\[
%    \boxed{\mathcal{R} \leq \left(\frac{\sqrt{T}}{\sqrt{2}-1}\right) \alpha \sqrt{T}}
%\]
%
%\bibliographystyle{unsrt}
%\bibliography{ref}
\end{document}
